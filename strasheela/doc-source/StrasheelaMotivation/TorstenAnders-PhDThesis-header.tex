
%% !! tmp: more debug info
%\setcounter{errorcontextlines}{\maxdimen}

%\usepackage{babel}
\usepackage[english]{babel} 
\usepackage[T1]{fontenc}
\usepackage[latin1]{inputenc}

\newcommand{\comment}[1]{}

\setlength\parskip{\medskipamount}
\setlength\parindent{0pt}


%% Zeilenabstand
\usepackage{setspace}
%\doublespacing
%\setstretch{1.5}

%% page margins as perscribed by QUB (does this already imply binding offset?)
\usepackage[lmargin=4cm,rmargin=2.5cm]{geometry}
%\usepackage[bindingoffset=1cm,lmargin=4cm,rmargin=2.5cm]{geometry}

%% to print a page presenting current page layout for testing
\usepackage{layouts}
\newcommand\showpage{%
  %\setlayoutscale{0.25}\setlabelfont{\tiny}%
  \setlayoutscale{0.5}\setlabelfont{\tiny}%
  \printheadingsfalse\printparametersfalse
  \currentpage\pagedesign}

%\usepackage{color}
\usepackage{graphicx}

\frenchspacing
% almost europeen fonts (etwa T1)
\usepackage{ae,aecompl}

\usepackage{amsmath}
\usepackage{amssymb}
%\usepackage{stmaryrd}

%% changes all fonts (incl. math)
%\usepackage{pxfonts}
%\usepackage{txfonts}

\usepackage{fancybox}           % for Bflushleft
\usepackage{fancyvrb}
%\DefineShortVerb{\+}
%\usepackage{verbatim}
%\newcommand{\code}[1]{\begin{Verbatim}#1\end{Verbatim}}

% \usepackage{algorithmicx}

\usepackage{booktabs}      

%% for rotating floats (environments sidewaysfigure and sidewaystable)
\usepackage{rotating}


%%
%% all copied and edited from clrscode.sty
%%
\newcommand{\TextHyphens}{\mathcode`\-=`\-\relax}
%\newcommand{\id}[1]{\ensuremath{\mathop{\mathit{\TextHyphens#1}}\nolimits}}
\newcommand{\id}[1]{\ensuremath{\mathit{\TextHyphens#1}}}
% Command for typesetting subarray ranges.
% \newcommand{\twodots}{\mathinner{\ldotp\ldotp}}


\comment{
%% fancy chapter heading with quote
\usepackage[avantgarde]{quotchap} 
\renewcommand\chapterheadstartvskip % quote and chap number on same level
             {\vspace*{-5\baselineskip}}
%% !! this changes all header fonts..
%\usepackage{helvet} % select Helvetica for title and quote
%\renewcommand\sectfont{\sffamily\bfseries}
}

%\usepackage{showkeys} % to debug references

%% defines commands like \vref, see Latex Companion pp. 68 ff
%% !! tmp comment and new def \vref to aviod rendering hassles
%\usepackage{varioref} 
\newcommand{\vref}[1]{\ref{#1}}

%% for BNF
\usepackage{syntax}
\setlength{\grammarparsep}{5.0pt plus1.0pt minus1.0pt} % default: 8.0pt plus1.0pt minus1.0pt
\setlength{\grammarindent}{10 em} % default: 2 em

%% to change appearance of upcode ' in \tt (to avoid confusing ` and ')
%\usepackage{upquote} 

%\usepackage[numbers,square,sort]{natbib}
\usepackage[authoryear,square]{natbib} % ,sort
%\usepackage{chicago}
\bibliographystyle{chicago}
%\bibliographystyle{plainnat}
\newcommand{\citeauthorp}[1]{[\citeauthor{#1}]}

\usepackage{float}
\floatstyle{boxed}
%\newfloat{BNF}{htb}{bnf}
\restylefloat{figure}
\floatplacement{figure}{htb} % H


%% defs \FloatBarrier, forcing float output before a certain point in the text
\usepackage[below,above]{placeins} 

%% I figure, listings package needs update to support Oz lang
% \usepackage{listings}
% \lstset{language=oz,
% %frame=single,
% numbers=left,numberstyle=\tiny,stepnumber=5, numbersep=5pt,
% columns=flexible, % fixed, fullflexible, 
% % basicstyle=\tt % no \bfseries for keywods
% basicstyle=\small\sf, % no keyword highlighting
% keywordstyle=\bfseries,
% %identifierstyle=\bfseries, % almost everything is a var in Oz...
% commentstyle=\emph,
% showstringspaces=false}


%\setcounter{tocdepth}{1}
%\setcounter{tocdepth}{5}        % tmp 
\setcounter{secnumdepth}{4}
\setcounter{tocdepth}{2}
%\setcounter{tocdepth}{3}
% \setcounter{tocdepth}{1}

% for pdftex only
% \pdfinfo{ 
%%   /Title (Constraint based computer assisted composition using the Oz programming model) 
%   %/Creator (TeX) 
%   %/Producer (pdfTeX 0.15a) 
%   /Author (Torsten Anders) 
%   %/CreationDate (D:19980212201000) 
%   %/ModDate (D:19980212201000) 
%   %/Subject (Example) 
%   /Keywords (Computer Assisted Composition, Constraint Programming) }

%% index creation
%\usepackage{makeidx}
%\makeindex


%% line numbers (used for formulas)

\comment{
% Verhindern von "Schusterjungen" und "Hurenkindern"
\clubpenalty = 10000
\widowpenalty = 10000
\displaywidowpenalty = 10000
\tolerance=500 %Zeilenumbruch
}


%%
%% customise description environment
%%
\renewcommand\descriptionlabel[1]%
  {\hspace{\labelsep}\textsf{#1}:} %% sans serif
%  {\hspace{\labelsep}\emph{#1}}

%%%%%%%%%%%%%%%%%%%%%%%%%%%%%% 
%%
%% Torstens Defs
%%

%% '@' is usually a LaTeX special char. The commands \makeatletter and \makeatother are used to mark a block of code where it is to be treated as 'normal'.

%\makeatletter

\newcommand{\code}[1]{\texttt{#1}}

\newcommand{\graphic}[3]
{\begin{figure}
    \centering
    %% !!?? Give the correct figure height and width in cm
    %% instead I am scaling the figure
    \includegraphics[scale=#2]{#1} #3
  \end{figure}}

\newenvironment{unfinished}{\begin{quote} {\bf unfinished:\\} \ttfamily}{\end{quote}}


%% 
%% for pretty printing logic relations etc. (some of these definition may actually never be needed and could be removed)
%%

% \DeclareMathOperator{\fun}{fun}

%% only within math mode: defines fun on single line
%\newcommand{\AnonymousFun}[2]{\lambda \: #1 \mathrel{.} #2}
%\newcommand{\AnonymousFun}[2]{\mathop{\lambda} #1 \mathrel{.} #2}
%\newcommand{\AnonymousFun}[2]{\lambda \: (#1) \mathrel{.} #2}
%\newcommand{\AnonymousFun}[2]{\fn \: (#1) \mapsto #2}
%\newcommand{\AnonymousFun}[2]{\fun \: #1 \mapsto #2}
% \newcommand{\AnonymousFun}[2]{\fun \: (#1) \mapsto #2}
%\newcommand{\AnonymousFun}[2]{#1 \mapsto #2}

%% only outside math mode
\newcommand{\Comment}[1]{{\footnotesize\sffamily\slshape /* #1 \hfill */}} %% \\
%\newcommand{\Comment}[1]{\ensuremath{text{\footnotesize\sffamily\slshape /* #1 \hfill */}}} %% \\
%\newcommand{\Comment}[1]{\ensuremath{\mathsf{/* #1 \hfill */}}} %% \\

%\newcommand{\MathComment}[1]{\mathrm{/* #1 \hfill */}} %% \\
%\newcommand{\MathComment}[1]{\parbox[t]{\textwidth}{/* #1 \hfill */}} 
%% optional arg is parbox width
%\newcommand{\MathComment}[1]{\mbox{\parbox{\footnotesize\sffamily\slshape /* #1 \hfill */}}} 
\newcommand{\MathComment}[2][\textwidth]{\parbox[t]{#1}{\footnotesize\sffamily\slshape /* #2 \hfill */}} 
%\newcommand{\AndNl}{\hfill $\wedge$\\}
\newcommand{\AndNl}{$\wedge$\\}
%\newcommand{\Def}{\ensuremath{\stackrel{\id{def}}{=}}}
\newcommand{\Def}{\ensuremath{\stackrel{\scriptscriptstyle\id{def}}{=}}}
%\newcommand{\Def}{\ensuremath{\equiv}}
%\newcommand{\Def}{\ensuremath{:=}} % spacing
%\newcommand{\Def}{\ensuremath{\triangleq}}
%\newcommand{\MyBox}[1]{\parbox[t]{\textwidth}{#1}}
%% !! Hoehe 8pt abhaengig von Zeichengroesse!!
%\newcommand{\MyBox}[1]{\raisebox{8pt-\height}{\shortstack[l]{#1}}}
%\newcommand{\MyBox}[2]{\begin{Bflushleft}[t]#1#2\end{Bflushleft}}
%% doc: usage requires '&' in each line
\newcommand{\MyBox}[2]{\ensuremath{\begin{aligned}[t]#1#2\end{aligned}}}
%% idea from L. Lamport (How to write a long formula)
%% doc: usage does _not_ require '&' 
% \newcommand{\MyBoxB}[2]{%
%   \ensuremath{%
%     \begin{array}[t]{@{}l@{\hspace*{.5em}}l@{}}
%       & \begin{array}[t]{@{}l@{}}
%         #1#2
%       \end{array}
%     \end{array}}}
%\newcommand{\MyBox}[2]{\ensuremath{\begin{array}[t]{l}#1#2\end{array}}}
%% \Apply{fun}{args}{following in last line} write fun call which allows nicely aligned line breaks in args 
%% fun is implicitly in math mode
%\newcommand{\Apply}[2]{$#1($ \parbox[t]{\textwidth}{#2$)$}}
%\newcommand{\Apply}[3]{\ensuremath{#1\MyBox{(#2)}{#3}}}
\newcommand{\Apply}[3]{\ensuremath{#1\MyBox{(#2)}{#3}}}
%\newcommand{\Apply}[3]{\ensuremath{#1\MyBox{(#2}{#3)}}}
%% \DefFun{fun(args)}{body}{following in last line}
%% fun(name) is implicitly in math mode
%\newcommand{\DefFun}[2]{\parbox[t]{\textwidth}{$#1 =$ \\ \Indent \parbox[t]{\textwidth}{#2}}}
%\newcommand{\DefFun}[3]{\ensuremath{\MyBox{\id{#1} \Def \: \\ \Indent \MyBox{#2}{#3}}}}
%\newcommand{\Defun}[3]{\ensuremath{\MyBox{&#1 \Def \: \\ &\Indent \MyBox{#2}{#3}}{}}}
\newcommand{\Defun}[3]{\ensuremath{#1 \Def \MyBox{#2}{#3}{}}}
\newcommand{\DefunB}[3]{\ensuremath{\MyBox{&#1 \Def \\&\Indent\MyBox{#2}{#3}{}}{}}}
%% variant for \DefFun for completeness for fun defs on a single line
%\newcommand{\DefunL}[2]{\ensuremath{\id{#1} \Def \:#2}}
%% creates a list which allows nicely aligned line breaks
%\newcommand{\List}[1]{$[$\parbox[t]{\textwidth}{#1}$]$}
\newcommand{\List}[2]{\ensuremath{\MyBox{[#1]}{#2}}}
%% creates a set which allows nicely aligned line breaks
\newcommand{\Set}[2]{\ensuremath{\MyBox{\{#1\}}{#2}}}
%% !! check amsmath for alternative formatting ways (skimmed doc: probably no easy better way..)
\newcommand{\DefPredicate}[3]{
  \begin{minipage}{\textwidth} 
    $\forall \: #2 :$ {\itshape\bfseries #1}$(#2) \leftrightarrow$ \\
    %% max breite ist breite der Seite, mit \\ umbruch moeglich
    \Indent\begin{minipage}{\textwidth}  
      #3
    \end{minipage}
  \end{minipage}
  \vspace{2 ex}}
% \newcommand{\Let}[2]{
%   $\exists \: #1$ : \\
%   \Indent\begin{minipage}{\textwidth}  
%     #2
%   \end{minipage}}
%% from L. Lamport (How to write a long formula)
\newcommand{\Let}[2]{%
  \ensuremath{%
    \begin{array}[t]{@{}l@{\hspace*{.5em}}l@{}}
      {\bf let} & \begin{array}[t]{@{}l@{}}
                    #1
                  \end{array}\\
      {\bf in} & #2
    \end{array}}}
%\newcommand{\Apply}[2]{$#1$(#2)}
% \newcommand{\If}[2]{
%   {\sffamily\bfseries if} $#1$ \\
%   %% max breite ist breite der Seite, mit \\ umbruch moeglich
%   \Indent\begin{minipage}{\textwidth}  
%     {\sffamily\bfseries then} #2
%   \end{minipage}}
% \newcommand{\IfElse}[3]{
%   {\sffamily\bfseries if} $#1$ \\
%   %% max breite ist breite der Seite, mit \\ umbruch moeglich
%   \Indent\begin{minipage}{\textwidth}  
%      {\sffamily\bfseries then} #2 \\
%      {\sffamily\bfseries else} #3
%   \end{minipage}}
%% in math mode (only functional use)
\newcommand{\IfL}[3]{\ensuremath{\mathbf{if} \: #1 \: \mathbf{then} \: #2 \: \mathbf{else} \: #3}}
\newcommand{\If}[3]{%
  \ensuremath{%
    \begin{array}[t]{@{}l@{\hspace*{.5em}}l@{}}
      {\bf if} & #1\\
      {\bf then} & #2\\
      {\bf else} & #3\\
    \end{array}}}
% \newcommand{\ForAll}[2]{$\forall \: #1 $ : \\
%   \Indent\begin{minipage}{\textwidth}  
%     #2
%   \end{minipage}} 
%% unfinished def
%\newcommand{\ForAll}[3]{\ensuremath{\MyBox{\forall \: #1 $ : \\ \MyBox{#2}{#3}}{}}
%\newcommand{\Symbol}[1]{\mbox{\sf #1}}
%\newcommand{\Sym}[1]{\mbox{\sf #1}} % alias for \Symbol
\newcommand{\Indent}{\hspace*{2 ex}}
\newcommand{\I}{\hspace*{2 ex}}
%% in math mode
\renewcommand{\And}{\ensuremath{\wedge}}
\newcommand{\AllTrue}{\ensuremath{\mathop{\bigwedge}}}
\newcommand{\Or}{\ensuremath{\vee}}
\newcommand{\Impl}{\ensuremath{\Rightarrow}}
\newcommand{\Equi}{\ensuremath{\Leftrightarrow}}
%\renewcommand{\Not}{\ensuremath{\neg}}

%\newcommand{\BooleanEq}{\stackrel{\mathrm{b}}{=}}
%\newcommand{\UnifyEq}{\stackrel{\mathrm{u}}{=}}

%% overwrite AMS colon def: simple punctuation sign
\renewcommand{\colon}{\ensuremath{\mathpunct{:}}}


%% fuer einen kl. Zeilenzwischenraum (in App. tables..)
%\newcommand{\minivspace}{\vspace{0.3 em}}
%% Zeilenumbruch mit etwas groesserem Zwischenraum danach
% \newcommand{\skippingNl}{\vspace{0.3 em}\\}



%\makeatother



%%%%%%%%%%%%%%%%%%%%%%%%%%%%%%

\hyphenation{Stra-shee-la}


\usepackage{url}
% must be the last package loaded
% draft disables hyperref
\usepackage{hyperref}



%%% Local Variables: 
%%% mode: latex
%%% TeX-master: "TorstenAnders-PhDThesis"
%%% End: 
